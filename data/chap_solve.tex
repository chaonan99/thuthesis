\chapter{车道汇合群决策模型求解}
\label{cha:solve}

在上一章中已经给出了模型的目标函数,并且提到需要对每辆车求解一个控制量输入$u_i(t)$,使得$J(\mathbf{u})$达到最小。注意到$u_t(t)$实际上是关于时间$t$的函数,$J(\mathbf{u})$实际上是函数的函数,数学上称为泛函。求泛函的极值需要用到变分法的相关知识。本章先对泛函与变分法进行简要说明,之后运用该方法求解不同情况下泛函的极值,得到决策控制方法。

\section{泛函与变分法}
泛函的概念最早在变分法中出现,并广泛应用于最优控制的相关问题中。

\subsection{泛函与泛函的变分}
抽象空间理论将泛函定义为向量空间上的实值函数,而在最优控制中往往限定为函数集上的实值函数。定义如下,
\begin{definition}[泛函]
函数集$\mathcal{U}(x)$,到实数集$\mathbb{R}$的映射$J: \mathcal{U}\rightarrow \mathbb{R}$称为函数集$\mathcal{U}$上的{\heiti 泛函},即,
\begin{equation}
J=J[u(x)], \ u(x)\in \mathcal{U}(x).
\end{equation}
\end{definition}
泛函的变分与函数的微分类似。首先需要定义函数的变分,定义如下,
\begin{definition}[函数的变分]
设$u(x), u_0(x) \in \mathcal{U}(x)$,{\heiti 函数的变分}定义为
\begin{equation}
\delta u(x)=u(x) - u_0(x)
\end{equation}
\end{definition}
由以上定义知,函数的变分仍是$x$的函数。另外根据泛函定义,函数集$\mathcal{U}(x)$应是向量空间,对加法和数乘封闭,因此$\delta u(x)\in \mathcal{U}(x)$。
之后定义泛函的连续性如下,
\begin{definition}[泛函的连续性]
$\forall \varepsilon > 0$,$\exists \delta > 0$,对于$\forall u(x), d(u,u_0)<\delta$有
\begin{equation}
\Delta J[u(x)]=J[u(x)+\delta u(x)]-J[u(x)] < \delta
\end{equation}
其中$d(u,u_0)$为函数空间$\mathcal{U}$上定义的某种范数,则称泛函$J[u(x)]$在范数$d$下,在$u_0(x)$连续。
\end{definition}
泛函的变分定义如下,
\begin{definition}[泛函的变分]
泛函的增量若能表示为
\begin{equation}
\Delta J[u(x)]=L[u(x),\delta u(x)] + \gamma[u(x), \delta u(x)],
\end{equation}
其中$L[u(x),\delta u(x)]$是$\delta u(x)$的线性连续泛函,$\gamma[u(x), \delta u(x)]$是$\delta u(x)$的高阶无穷小,则称$L[u(x),\delta u(x)]$为泛函$J[u(x)]$的变分,记为
\begin{equation}
\delta J = L[u(x),\delta u(x)]
\end{equation}
\end{definition}
关于线性连续泛函的概念,以及泛函变分更严谨的定义,参见泛函分析的教材\cite{PETERD2007Functional}。

\subsection{变分法}
求泛函极值的方法称为变分法。泛函极值的定义与函数极值类似。下面不加证明地给出如下定理:
\begin{theorem}
若泛函$J[u(x)]$在$u_0(x)$有极值,则
\begin{equation}
\delta J[u_0(x)]=0
\end{equation}
\end{theorem}
注意,上述定理只是泛函取极值的必要条件。至于是否存在极值,是极大还是极小值,需要根据实际问题的性质确定。

变分学中有如下三类基本问题。三者区别在于泛函形式不同,而目标都是求泛函极小值。
\paragraph{拉格朗日(Lagrange)问题} 泛函形式为
\begin{equation}
J = \int_{t_0}^{t_\mathrm{f}}F(t,x,\dot{x})\mathrm{d}t,
\end{equation}
其中$t_0$为初始时间,$t_f$为终端时间。
\paragraph{梅耶(Mayer)问题} 泛函形式为
\begin{equation}
J=\theta[x_\mathrm{f},t_\mathrm{f}].
\end{equation}
该泛函是终端时间$t_\mathrm{f}$和终端函数$x_\mathrm{f}=x(t_mathrm{f})$的函数。
\paragraph{波尔查(Bolza)问题} 泛函形式为
\begin{equation}
J=\theta[x_\mathrm{f},t_\mathrm{f}]+\int_{t_0}^{t_\mathrm{f}}F(t,x,\dot{x})\mathrm{d}t.
\end{equation}
可见,前两者是后者的特殊形式。

\subsection{最优控制的必要条件}
最优控制问题常写作如下形式的波尔查问题
\begin{equation}
J(\bm{u})=\theta[t_\mathrm{f},\bm{x}_\mathrm{f}]+\int_{t_0}^{t_\mathrm{f}}L[\bm{x}(t),\bm{u}(t),t]\mathrm{d}t,
\end{equation}
其状态方程和初始状态为
\begin{equation}
\dot{\bm{x}}=\bm{f}[\bm{x}(t),\bm{u}(t),t], \quad \bm{x}(t_0)=\bm{x}_0.
\end{equation}
定义{\heiti 哈密顿函数}为
\begin{equation}
H(\bm{x},\bm{u},\bm{\lambda},t)=L(\bm{x},\bm{u},t)+\bm{\lambda}^\mathrm{T}\bm{f}(\bm{x},\bm{u},t),
\end{equation}

下面针对上述波尔查问题的形式,给出几种情况下的最优控制必要条件。若最优控制的损失函数退化为拉格朗日或梅耶问题的形式,其必要条件也可以相应得出。
\\
\\
\begin{enumerate}
\item {\heiti 终端自由,$t_\mathrm{f}$给定的情形}
\\ 最优控制的必要条件为
\begin{gather}
\frac{\partial H}{\partial\bm{u}}=0,\label{eq:nc:control}\\
\dot{\bm{u}}=-\frac{\partial H}{\partial\bm{x}},\label{eq:nc:company}\\
\bm{\lambda}(t_\mathrm{f})=\left. \frac{\partial \theta}{\partial \bm{x}}\right|_{t_\mathrm{f}},\\
\dot{\bm{x}}=\bm{f}(\bm{x},\bm{u},t),\\
\bm{x}(t_0)=\bm{x}_0.\label{eq:nc:last}
\end{gather}
其中式\ref{eq:nc:control}也被称作{\heiti 控制方程},式\ref{eq:nc:company}也被称作{\heiti 伴随方程}。若退化为拉格朗日问题,则$\theta[t_\mathrm{f},\bm{x}_\mathrm{f}]\equiv 0$,式\ref{eq:nc:company}变为
\begin{equation}
\bm{\lambda}(t_\mathrm{f})=0.
\end{equation}

\item {\heiti 终端受限,$t_\mathrm{f}$给定的情形}
\\设终端约束条件为
\begin{equation}
\bm{g}(\bm{x}_\mathrm{f},t_\mathrm{f})=0,
\end{equation}
最优控制的必要条件为
\begin{gather}
\frac{\partial H}{\partial\bm{u}}=0,\label{eq:cc:control}\\
\dot{\bm{u}}=-\frac{\partial H}{\partial\bm{x}},\label{eq:cc:company}\\
\bm{\lambda}(t_\mathrm{f})=\left. \frac{\partial \theta}{\partial \bm{x}}\right|_{t_\mathrm{f}},\\
\dot{\bm{x}}=\bm{f}(\bm{x},\bm{u},t),\\
\bm{x}(t_0)=\bm{x}_0.\label{eq:cc:last}
\end{gather}

\item {\heiti 终端受限,$t_\mathrm{f}$未给定的情形}
\\最优控制的必要条件为
\begin{gather}
\frac{\partial H}{\partial\bm{u}}=0,\label{eq:cc:control}\\
\dot{\bm{u}}=-\frac{\partial H}{\partial\bm{x}},\label{eq:cc:company}\\
\bm{\lambda}(t_\mathrm{f})=\left. \frac{\partial \theta}{\partial \bm{x}}\right|_{t_\mathrm{f}},\\
\dot{\bm{x}}=\bm{f}(\bm{x},\bm{u},t),\\
\bm{x}(t_0)=\bm{x}_0.\label{eq:cc:last}
\end{gather}

\end{enumerate}