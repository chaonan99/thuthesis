\thusetup{
  %******************************
  % 注意:
  %   1. 配置里面不要出现空行
  %   2. 不需要的配置信息可以删除
  %******************************
  %
  %=====
  % 秘级
  %=====
  secretlevel={秘密},
  secretyear={10},
  %
  %=========
  % 中文信息
  %=========
  ctitle={基于车路协同的车道汇合\\群决策问题研究},
  cdegree={工学硕士},
  cdepartment={~~自~动~化~系},
  cmajor={~~自~动~化},
  cauthor={~~陈~昊~楠},
  csupervisor={~~张~~~~~~毅\hspace*{16pt}教~授},
  % cassosupervisor={陈文光教授}, % 副指导老师
  % ccosupervisor={某某某教授}, % 联合指导老师
  % 日期自动使用当前时间,若需指定按如下方式修改:
  % cdate={超新星纪元},
  %
  % 博士后专有部分
  cfirstdiscipline={计算机科学与技术},
  cseconddiscipline={系统结构},
  postdoctordate={2009年7月——2011年7月},
  id={编号}, % 可以留空: id={},
  udc={UDC}, % 可以留空
  catalognumber={分类号}, % 可以留空
  %
  %=========
  % 英文信息
  %=========
  etitle={An Introduction to \LaTeX{} Thesis Template of Tsinghua University v\version},
  % 这块比较复杂,需要分情况讨论:
  % 1. 学术型硕士
  %    edegree:必须为Master of Arts或Master of Science(注意大小写)
  %             “哲学、文学、历史学、法学、教育学、艺术学门类,公共管理学科
  %              填写Master of Arts,其它填写Master of Science”
  %    emajor:“获得一级学科授权的学科填写一级学科名称,其它填写二级学科名称”
  % 2. 专业型硕士
  %    edegree:“填写专业学位英文名称全称”
  %    emajor:“工程硕士填写工程领域,其它专业学位不填写此项”
  % 3. 学术型博士
  %    edegree:Doctor of Philosophy(注意大小写)
  %    emajor:“获得一级学科授权的学科填写一级学科名称,其它填写二级学科名称”
  % 4. 专业型博士
  %    edegree:“填写专业学位英文名称全称”
  %    emajor:不填写此项
  edegree={Doctor of Engineering},
  emajor={Computer Science and Technology},
  eauthor={Xue Ruini},
  esupervisor={Professor Zheng Weimin},
  eassosupervisor={Chen Wenguang},
  % 日期自动生成,若需指定按如下方式修改:
  % edate={December, 2005}
  %
  % 关键词用“英文逗号”分割
  % ckeywords={\TeX, \LaTeX, CJK, 模板, 论文},
  % ekeywords={\TeX, \LaTeX, CJK, template, thesis}
  ckeywords={车路协同, 车道汇合, 最优控制},
  ekeywords={VIC, CAV, merging lane, optimal control}
}

% 定义中英文摘要和关键字
\begin{cabstract}
  % 论文的摘要是对论文研究内容和成果的高度概括。摘要应对论文所研究的问题及其研究目
  % 的进行描述,对研究方法和过程进行简单介绍,对研究成果和所得结论进行概括。摘要应
  % 具有独立性和自明性,其内容应包含与论文全文同等量的主要信息。使读者即使不阅读全
  % 文,通过摘要就能了解论文的总体内容和主要成果。

  % 论文摘要的书写应力求精确、简明。切忌写成对论文书写内容进行提要的形式,尤其要避
  % 免“第 1 章……;第 2 章……;……”这种或类似的陈述方式。

  智能车(Connected Automated Vehicle, CAV)技术在最近几年迅速发展,其控制理论的研究也日益得到重视。智能车可以通过车路协同平台(Vehicle Infrastructure Cooperating System, VICS)进行车辆之间、车辆与道路设施的信息交换,为提高整体交通运行效率和安全性提供了可能。本文以道路交通中常见的车道汇合口为研究场景,研究多智能车在该场景下的群决策问题。具体地,本文运用系统状态方程对车道汇合场景进行建模,模型同时考虑了安全性约束与最小化控制量输入的目标,并先后运用最优控制、约束优化的数值解法和高次方程组的数值解法,对问题进行一步步的简化,最终将一个NP难搜索问题的复杂度大大减小,在数值计算软件中可以较为容易地求解。

  智能车群决策问题中,通行顺序的确定是一个重要课题。大部分文献都采用了先进先出(First-in first-out, FIFO)的通行顺序,但该顺序在交汇口有主路辅路的优先级区分时效率较低。本文结合最优控制中解得的控制量形式,提出了一种基于最优通行时间的顺序决策算法,能够明显提高通行效率,能够在辅路车速明显低于主路的情况下获得应用。

  另外,本文还自主研发了一套仿真软件和相应的应用程序,并将其公开到网络,为之后的相关研究提供了帮助。

  % 本文介绍清华大学论文模板 \thuthesis{} 的使用方法。本模板符合学校的本科、硕士、
  % 博士论文格式要求。

  % 本文的创新点主要有:
  % \begin{itemize}
  %   \item 用例子来解释模板的使用方法;
  %   \item 用废话来填充无关紧要的部分;
  %   \item 一边学习摸索一边编写新代码。
  % \end{itemize}

  % 关键词是为了文献标引工作、用以表示全文主要内容信息的单词或术语。关键词不超过 5
  % 个,每个关键词中间用分号分隔。(模板作者注:关键词分隔符不用考虑,模板会自动处
  % 理。英文关键词同理。)
\end{cabstract}

% 如果习惯关键字跟在摘要文字后面,可以用直接命令来设置,如下:
% \ckeywords{\TeX, \LaTeX, CJK, 模板, 论文}

\begin{eabstract}
  Connected Automated Vehicle (CAV) technology has developed rapidly in recent years, and the corresponding control theory has been attracting more and more attention. With the help of Vehicle Infrastructure Cooperating System (VICS), CAVs are able to exchange information with other CAVs and infrastructures on the road, bringing up the possibility of further improvement on the overall efficiency and safety of the transportation network. This paper address the problem of multiple CAVs group decision making under the common circumstances of mering roadways. Concretely, we present a model using system state equation, taken both the safety constraint and the minimization of control input into consideration. We simplify the model step by step using optimal control theory, the constrained optimization with numerical solution and the higher order equations, which ultimately change the NP-Hard nature of the original problem, making it easily solved by software for numerical calculation.

  The traffic order is a crucial topic in the group decision making process of CAVs. Most of the literature uses the first-in, first-out (FIFO) order, which is less efficient when the priority of the primary route is different at the merging zone. In this paper, a sequential decision algorithm based on optimal passing time is proposed in this paper, which can effectively improve the traffic efficiency. This can be applied to the circumstances when the speed on the auxiliary road is significantly lower than that on the main road

  In addition, a set of simulation software and the correspond application are also developed from scratch and made publicly available for the benefit of future research.

   % Key words are terms used in a dissertation for indexing, reflecting core
   % information of the dissertation. An abstract may contain a maximum of 5 key
   % words, with semi-colons used in between to separate one another.
\end{eabstract}

% \ekeywords{\TeX, \LaTeX, CJK, template, thesis}
