\chapter{总结与展望}
\label{cha:sum}

\section{总结}
本文运用系统状态方程对车道汇合场景进行建模,模型同时考虑了安全性约束与最小化控制量输入的目标,提出了一种基于最优控制理论的求解方法。该方法的主要创新之处在于对终端确定,状态量和控制量都存在约束情形的求解。该情形下,最优控制理论仅给出了最优控制的必要条件,而无法给出解析解。本文在利用必要条件解出控制量的形式并参数化后,建立单车控制系统的约束优化问题,对简化后的系统进行搜索,用数值方法给出了最优控制。进一步的,本文还在实验中发现了最优控制下控制量连续以及其他对称性,进一步将最优控制的求解转化为方程的求解,将一个NP难的搜索问题变得可解。

另外,本文还针对先进先出顺序的弊端,结合在终端自由情形下,最优控制给出的最优通行时间进行排序,该方法能显著提高通行效率,在辅路速度明显低于主路的常见路况中有巨大的应用前景。

本文还自主研发了基于{\ttfamily Python}的仿真系统,所有代码均公开于网络,为之后的研究提供了便利。

\section{未来工作}
为了使本文提出的算法真正走向实用,还需要解决的相关问题,以及解决的思路如下。
\begin{enumerate}[label=(\arabic*), wide=\parindent]
% \begin{enumerate}
\item 在单车的最优控制求解中,没有考虑交通非常拥堵,需要停车等待的问题。停车等待在高速公路的匝道口不太常见,但在城市道路等低速环境中很可能发生。控制算法应考虑到这种情况。
\item 控制区不碰撞条件需要证明。本文提出的安全性约束仅针对进入交汇区的时刻,在控制区的碰撞避免还需进一步讨论。事实上,当同车道所有车辆以相同速度进入控制区,且车辆都具有相同约束条件时,在本文提出的最优控制函数下,其间距是单调减小的,此时只要满足假设\ref{ass:restrict},车辆在交汇口不碰撞,则在控制区也不会碰撞。而车辆初始速度不同时,可以增设一个安全性增强区,对车辆的车速进行调整,即可解决控制区的安全性问题。
\item 本文提出的最优控制求解算法分了多种情况,算法流程比较复杂,在实际情况中可能无法实时求解。这可以通过建立一个映射表的方法。对本文假设的情景来说,控制区长度$L$,期望速度$v_\mathrm{d}$,最大速度$v_{\max}$,最大加速度$u_{\max}$等参数都是固定的,初始速度$v_0$的选择一般也比较少,仅通行时间$t_\mathrm{m}$变化较大。可以建立一张表,将不同$v_0$与$t_\mathrm{m}$对应的控制策略保存在表中,即可快速获得所需控制函数。
\end{enumerate}

仿真方面,本文开发的仿真软件还未完全将提出的最优控制算法进行整合,代码还需要进一步完善。

另外,本文的算法框架还可以在其他场景中进行扩展。首先,本文的建模和求解方法可以在交叉路口得到应用。其次,本文的算法框架容易扩展到通信有延时的情况中,因为算法需要交互的信息量很少,只需要知道前一辆通过车辆的$t_\mathrm{m}$值。不过该值依赖于车辆之间的时钟同步,需要进一步研究通信延时的影响。另外,将算法扩展到混合通行的场景中也是一个有趣且很有前景的研究方向。